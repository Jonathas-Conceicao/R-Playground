\documentclass[11pt]{scrartcl} % Font size
%%%%%%%%%%%%%%%%%%%%%%%%%%%%%%%%%%%%%%%%%
% Wenneker Assignment
% Structure Specification File
% Version 2.0 (12/1/2019)
%
% This template originates from:
% http://www.LaTeXTemplates.com
%
% Authors:
% Vel (vel@LaTeXTemplates.com)
% Frits Wenneker
%
% License:
% CC BY-NC-SA 3.0 (http://creativecommons.org/licenses/by-nc-sa/3.0/)
% 
%%%%%%%%%%%%%%%%%%%%%%%%%%%%%%%%%%%%%%%%%

%----------------------------------------------------------------------------------------
%	PACKAGES AND OTHER DOCUMENT CONFIGURATIONS
%----------------------------------------------------------------------------------------

\usepackage{amsmath, amsfonts, amsthm} % Math packages

\usepackage{listings} % Code listings, with syntax highlighting

%% \usepackage[english]{babel} % English language hyphenation
\usepackage[portuguese]{babel}

\usepackage{graphicx} % Required for inserting images
\graphicspath{{Figures/}{./}} % Specifies where to look for included images (trailing slash required)

\usepackage{booktabs} % Required for better horizontal rules in tables

\numberwithin{equation}{section} % Number equations within sections (i.e. 1.1, 1.2, 2.1, 2.2 instead of 1, 2, 3, 4)
\numberwithin{figure}{section} % Number figures within sections (i.e. 1.1, 1.2, 2.1, 2.2 instead of 1, 2, 3, 4)
\numberwithin{table}{section} % Number tables within sections (i.e. 1.1, 1.2, 2.1, 2.2 instead of 1, 2, 3, 4)

\setlength\parindent{0pt} % Removes all indentation from paragraphs

\usepackage{enumitem} % Required for list customisation
\setlist{noitemsep} % No spacing between list items

%----------------------------------------------------------------------------------------
%	DOCUMENT MARGINS
%----------------------------------------------------------------------------------------

\usepackage{geometry} % Required for adjusting page dimensions and margins

\geometry{
	paper=a4paper, % Paper size, change to letterpaper for US letter size
	top=2.5cm, % Top margin
	bottom=3cm, % Bottom margin
	left=3cm, % Left margin
	right=3cm, % Right margin
	headheight=0.75cm, % Header height
	footskip=1.5cm, % Space from the bottom margin to the baseline of the footer
	headsep=0.75cm, % Space from the top margin to the baseline of the header
	%showframe, % Uncomment to show how the type block is set on the page
}
\setlength{\parskip}{\baselineskip}%
\setlength{\parindent}{0pt}%

%----------------------------------------------------------------------------------------
%	FONTS
%----------------------------------------------------------------------------------------

\usepackage[utf8]{inputenc} % Required for inputting international characters
\usepackage[T1]{fontenc} % Use 8-bit encoding

\usepackage{fourier} % Use the Adobe Utopia font for the document

%----------------------------------------------------------------------------------------
%	SECTION TITLES
%----------------------------------------------------------------------------------------

\usepackage{sectsty} % Allows customising section commands

\sectionfont{\vspace{6pt}\centering\normalfont\scshape} % \section{} styling
\subsectionfont{\normalfont\bfseries} % \subsection{} styling
\subsubsectionfont{\normalfont\itshape} % \subsubsection{} styling
\paragraphfont{\normalfont\scshape} % \paragraph{} styling

%----------------------------------------------------------------------------------------
%	HEADERS AND FOOTERS
%----------------------------------------------------------------------------------------

\usepackage{scrlayer-scrpage} % Required for customising headers and footers

\ohead*{} % Right header
\ihead*{} % Left header
\chead*{} % Centre header

\ofoot*{} % Right footer
\ifoot*{} % Left footer
\cfoot*{\pagemark} % Centre footer
 % Include the file specifying the document structure and custom commands
\usepackage{multirow}

\title{
  \normalfont\normalsize
  \textsc{Universidade Federal de Pelotas, Ciência da Computação}\\ % Your university, school and/or department name(s)
  \vspace{25pt} % Whitespace
  \rule{\linewidth}{0.5pt}\\ % Thin top horizontal rule
  \vspace{20pt} % Whitespace
  {\huge Coléta e Análise de Dados Estatísticos}\\ % The assignment title
  \vspace{12pt} % Whitespace
  \rule{\linewidth}{2pt}\\ % Thick bottom horizontal rule
  \vspace{12pt} % Whitespace
}

\author{
  \LARGE Jonathas Conceição\\
  \LARGE Juan Rios
}

\date{\normalsize\today} % Today's date (\today) or a custom date

\begin{document}

\maketitle % Print the title

%----------------------------------------------------------------------------------------

\section{Introdução}

Para que possamos entender melhor a utilização da estatística em nosso dia a dia ou até mesmo em nossas respectivas áreas, é importante a realização de trabalhos que possam nos proporcionar tais demonstrações, visando compreender a sua importância e complexidade. Devido a isto, foi proposto como trabalho a realização de um levantamento estatístico por meio de um questionário, onde cada entrevistado deveria responder uma série de questões relacionadas ao curso vigente, a UFPel e até mesmo algumas questões um pouco mais pessoais, como o consumo de álcool. Então, após a conclusão da etapda de coleta de dados escolhemos seis variáveis do total de quarenta e duas possivéis, as variáveis que escolhemos para a análise estatística foram:

\paragraph{\textbf{Curso:}} Para que possamos saber quais cursos participaram da etapa de lavatamento de dados.

\paragraph{\textbf{Consome álcool?}} Tentar relacionar o consumo de álcool com alguma possível influência na alteração dos resultados de uma pessoa, como o número de reprovações, desistência do curso, dentre outros.

\paragraph{\textbf{Pratica alguma atividade física?}} Demonstração da influência na obtenção de resultados seja eles bons ou ruins, pelo meio da prática de alguma atividade física juntamente com a atividade acadêmica.

\paragraph{\textbf{Tem alguma prática espiritual/filosófica ou religião?}} Novamente, tentar relacionar a prática espiritual ou filosófica como meio de influência no comportamento do entrevistado.

\paragraph{\textbf{Tempo médio de uso diário do Smartphone:}} A influência do uso excessivo do Smartphone no comportamento do entrevistado.

\paragraph{\textbf{Quantas disciplinas você já cursou (incluindo as reprovações e os aproveitamentos):}} Saber qual a influência no rendimento do entrevistado com número de disciplinas cursadas conforme ela aumenta ou diminui.

\paragraph{\textbf{Já pensou em desistir do curso?}} O quanto a resposta dessa questão é influência pelas demais perguntas.  

\paragraph{\textbf{Qual seu nível de satisfação com a infraestrutura da UFPel?}} Buscar entender o nível de satisfação dos entrevistado com relação a universidade e tentar relacionar o peso de suas respostas com  relação as outras questões respondidas.

Após toda a coleta de dados o objetivo é a apresentação de gráficos para que seja possível uma melhor visualização das amostras coletadas. Neste trabalho apresentamos então dois conjuntos de gráficos, na Seção Gráficos diretos apresentamos gráficos gerados diretamente dos dados coletados,
já na Seção Gráficos cruzados, nós apresentamos gráficos crusando algumas informações dos dados de entrada.

\clearpage

%----------------------------------------------------------------------------------------

\section{Gráficos diretos}

Nessa Seção será demonstrada os gráficos diretos, ou seja, gerados diretamente dos dados coletados sem nenhum tipo de tratamento ou semântica sobre eles, apenas medidas de tendência central como: média, mediana e etc.

\subsection{Curso}

\begin{figure}[h]
  \centering
  \includegraphics[page=1, width=1.0\columnwidth]{Graphs.pdf}
  \caption{Distribuição do curso dos alunos}
  \label{fig:graph1}
\end{figure}

Gráfico que demonstra a distribuição dos cursos, onde podem ser observado que o maior número de entrevistados encontra-se nos cursos de computação. Sendo o com maior volume de dados na Ciência da Computação, como pode ser observado na figura \ref{fig:graph1}.

\clearpage

\subsection{Semestre de Ingresso}

\begin{figure}[h]
  \centering
  \includegraphics[page=2, width=1.0\columnwidth]{Graphs.pdf}
  \label{fig:graph2}
  \caption{Ano/semestre de ingresso dos alunos}
\end{figure}

Gráfico que demonstra a distribuição dos ingresso dos entrevistados. Há uma grande predominância dos ingressantes do semestre 2017/2, porém, vale a pena observar que entre os períodos de 2013/2 e 2019/1 (excluíndo 2017/2, 2015/1 e 2016/2) a variância de alunos é praticamente nula. É possível também avavaliar que a maioria dos entrevistados encontram-se pelo 5º semestre de seu curso. Figura \ref{fig:graph2}.

\clearpage

\subsection{Consumo de álcool?}

\begin{figure}[h]
  \centering
  \includegraphics[page=3, width=1.0\columnwidth]{Graphs.pdf}
  \label{fig:graph3}
  \caption{Distribuição do consumo de álcool}
\end{figure}

Gráfico que demonstra o consumo de álcool dos entrevistados. Não há muito a observar, apenas destacar que a maioria da amostra ingere álcool de forma ocasional.  Figura \ref{fig:graph3}.

\clearpage

\subsection{Pratica alguma atividade física?}

\begin{figure}[h]
  \centering
  \includegraphics[page=4, width=1.0\columnwidth]{Graphs.pdf}
  \label{fig:graph4}
  \caption{Distribuição da pratica de atividade física}
\end{figure}

Gráfico que demonstra a pratica, ou não, de alguma atividade física dos entrevistados. Neste gráfico podemos oberservar que apenas 37.5\% dos entrevistados não praticam alguma atividade física e mais da metada (62.5\%) deles praticam alguma atividade. Figura \ref{fig:graph4}.

\clearpage

\subsection{Prática espiritual, filosófica ou religião}

\begin{figure}[h]
  \centering
  \includegraphics[page=5, width=1.0\columnwidth]{Graphs.pdf}
  \label{fig:graph5}
  \caption{Distribuição da prática espiritual, filosófica ou religiosa}
\end{figure}

Gráfico que demonstra a pratica, ou não, de alguma atividade espititual, filosófica ou religiosa dos entrevistados. É possível observar que mais da metade dos entrevistados possuí alguma das práticas (60.22\%) e apenas 39.7\% não. Figura \ref{fig:graph5}.

\clearpage

\subsection{Tempo médio do uso diário do smartphone}

\begin{figure}[h]
  \centering
  \includegraphics[page=6, width=1.0\columnwidth]{Graphs.pdf}
  \label{fig:graph6}
  \caption{Tempo médio do uso diário do smartphone}
\end{figure}

Gráfico que demonstra o tempo médio do uso de smartphone, podemos observar quase que um gráfico assimétrico positivo para direita, onde a maioria dos entrevistados usa em média o smartphone de 4 a 5 horas por dia. Figura \ref{fig:graph6}.

\clearpage

\subsection{Número de disciplinas cursadas}

\begin{figure}[h]
  \centering
  \includegraphics[page=7, width=1.0\columnwidth]{Graphs.pdf}
  \label{fig:graph7}
  \caption{Número de disciplanas cursadas}
\end{figure}

Gráfico que demonstra o número de disciplinas cursadas pelos entrevistados. Há uma grande dispersão nos dados, porém, a maioria concentra-se entre 15 a 40 cadeiras. Figura \ref{fig:graph7}.

\clearpage

\subsection{Já pensou em desistir do curso?}

\begin{figure}[h]
  \centering
  \includegraphics[page=8, width=1.0\columnwidth]{Graphs.pdf}
  \label{fig:graph8}
  \caption{Já pensou em desitir do curso}
\end{figure}

Gráfico que demonstra se os entrevistados já pensaram em desistir dos seus atuais cursos. O que pode ser comentado a respeito é que mais da metade dos entrevistados já pensou em desistir do seu curso. Figura \ref{fig:graph8}.

\clearpage

\subsection{Qual seu nível de satisfação com a infraestrutura da UFPel?}

\begin{figure}[h]
  \centering
  \includegraphics[page=9, width=1.0\columnwidth]{Graphs.pdf}
  \label{fig:graph9}
  \caption{Nível de satisfação com a infraestrutura da UFPel}
\end{figure}

Gráfico que demonstra a satisfação com a infraestrutrau da UFPel. Apresentou um nível satisfatório entre os entrevistados, variando de 5 até 8 o nível de satisfação. Figura \ref{fig:graph9}.

\clearpage

%----------------------------------------------------------------------------------------

\section{Inferência de valores}

Com os valores das amostras coletadas podemos estimar valores da população,
nesta Seção apresentamos na Tabela \ref{tab1} as médias dos valores numéricos estimados para população.
Todas as medidas foram calculadas sobre um nível de confiança de $95\%$ e
usando a Distribuição \textit{t de Student}.

\begin{table}[h]
\centering
\begin{tabular}{l|c}
\hline
\multicolumn{1}{c|}{Variável}                                                              & Intervalo              \\ \hline
\begin{tabular}[c]{@{}l@{}}Tempo médio de uso diário\\ do smartphone em horas\end{tabular} & {[}4.3931; 5.3796{]}   \\ \hline
Número de disciplinas cursadas                                                             & {[}27.7199; 32.4505{]} \\ \hline
\begin{tabular}[c]{@{}l@{}}Nível de satisfação com\\ a estrutura da UFPel\end{tabular}     & {[}6.4821; 7.0065{]}   \\ \hline
\end{tabular}
\caption{Intervalo da média da população sobre um intervalo de confiança de $95\%$}
\label{tab1}
\end{table}

\begin{table}[h]
\centering
\begin{tabular}{l|c|c|c}
\hline
\multicolumn{1}{c|}{Variável}                                                                               & Media Estimada      & Valor-P                  & Conclusão                                                                                    \\ \hline
\multirow{2}{*}{\begin{tabular}[c]{@{}l@{}}Tempo médio de uso diário\\ do smartphone em horas\end{tabular}} & Ciência: 4.7926     & \multirow{2}{*}{0.9233}  & \multirow{2}{*}{\begin{tabular}[c]{@{}c@{}}Há diferença real\\ entre as médias\end{tabular}} \\ \cline{2-2}
                                                                                                            & Engenharia: 4.8478  &                          &                                                                                              \\ \hline
\multirow{2}{*}{\begin{tabular}[c]{@{}l@{}}Número de disciplinas cursadas                \end{tabular}}     & Ciência: 32.2561    & \multirow{2}{*}{0.04849} & \multirow{2}{*}{\begin{tabular}[c]{@{}c@{}}Há diferença real\\ entre as médias\end{tabular}} \\ \cline{2-2}
                                                                                                            & Engenharia: 26.8913 &                          &                                                                                              \\ \hline
\multirow{2}{*}{\begin{tabular}[c]{@{}l@{}}Nível de satisfação com\\ a estrutura da UFPel\end{tabular}}     & Ciência: 6.5853     & \multirow{2}{*}{0.1839}  & \multirow{2}{*}{\begin{tabular}[c]{@{}c@{}}Há diferença real\\ entre as médias\end{tabular}} \\ \cline{2-2}
                                                                                                            & Engenharia: 7.0000  &                          &                                                                                              \\ \hline
\end{tabular}
\caption{Comparação das médias de alunos da Ciência e Engenharia da Computação}
\label{tab2}
\end{table}

\clearpage

%----------------------------------------------------------------------------------------

\section{Gráficos cruzados}

Nessa Seção foi cruzado alguns dados para que seja demonstrado uma possível relação entre eles e assim conseguir inferir melhores resultados do que os dos gráficos diretos.

\subsection{Já pensou em desistir (por Curso)}

\begin{figure}[h]
  \centering
  \includegraphics[page=10, width=1.0\columnwidth]{Graphs.pdf}
  \label{fig:graph10}
  \caption{Distribuição de desistência por curso}
\end{figure}

Gráfico que demonstra o cruzamento dos dados do nome do curso com a desistência. É possível notar que devido a maior amostra de dados serem dos cursos de computção eles são os que mais é possível inferir algum tipo de resultado sobre os dados, sendo assim na Ciência da é o que aprenseta o maior número de desistência, passando um pouco mais da metade do entrevistados. \ref{fig:graph10}.

\clearpage

\subsection{Quantas disciplinas já cursou e já pensou em desistir do curso}

\begin{figure}[h]
  \centering
  \includegraphics[page=11, width=1.0\columnwidth]{Graphs.pdf}
  \label{fig:graph11}
  \caption{Distrubuição de disciplinas cursadas e desistência}
\end{figure}

Gráfico que demonstra o cruzamento dos dados das disciplinas crusadas com a desistência do curso atual. É possível notar que há pouca variação nas respostas com relação a desistência, ou não, do curso atual, e que esses valores quase se aproximam demonstrando que para essa amostra coletada o fato de ela ter cursado menos ou mais cadeiras não é algo de extrema relevancia para a decisão. Figura \ref{fig:graph11}.

\clearpage

\subsection{Uso do Smartphone com pratica de atividades físicas}

\begin{figure}[h]
  \centering
  \includegraphics[page=12, width=1.0\columnwidth]{Graphs.pdf}
  \label{fig:graph12}
  \caption{Uso do smartphone e pratica de atividade física}
\end{figure}

Gráfico que demonstra o cruzamento dos dados do uso do smartphone com a pratica de atividades físicas. O mais notável é que aqueles que praticam atividades físicas tendem a passar menos tempo usando o smartphone, variando entre 2 a 3 horas de uso diário do dispositivo. Fgiura \ref{fig:graph12}.

\clearpage

\subsection{Satisfação com a estrutura da UFPel e consumo de álcool}

\begin{figure}[h]
  \centering
  \includegraphics[page=13, width=1.0\columnwidth]{Graphs.pdf}
  \label{fig:graph13}
  \caption{Satisfação com a estrutura da UFPel e o consumo de álcool}
\end{figure}

Gráfico que demonstra o cruzamento dos dados da satisfação com a estrutura da UFPel e o consumo de álcool. Figura \ref{fig:graph13}.

\end{document}
