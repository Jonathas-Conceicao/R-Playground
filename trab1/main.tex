\documentclass[11pt]{scrartcl} % Font size
%%%%%%%%%%%%%%%%%%%%%%%%%%%%%%%%%%%%%%%%%
% Wenneker Assignment
% Structure Specification File
% Version 2.0 (12/1/2019)
%
% This template originates from:
% http://www.LaTeXTemplates.com
%
% Authors:
% Vel (vel@LaTeXTemplates.com)
% Frits Wenneker
%
% License:
% CC BY-NC-SA 3.0 (http://creativecommons.org/licenses/by-nc-sa/3.0/)
% 
%%%%%%%%%%%%%%%%%%%%%%%%%%%%%%%%%%%%%%%%%

%----------------------------------------------------------------------------------------
%	PACKAGES AND OTHER DOCUMENT CONFIGURATIONS
%----------------------------------------------------------------------------------------

\usepackage{amsmath, amsfonts, amsthm} % Math packages

\usepackage{listings} % Code listings, with syntax highlighting

%% \usepackage[english]{babel} % English language hyphenation
\usepackage[portuguese]{babel}

\usepackage{graphicx} % Required for inserting images
\graphicspath{{Figures/}{./}} % Specifies where to look for included images (trailing slash required)

\usepackage{booktabs} % Required for better horizontal rules in tables

\numberwithin{equation}{section} % Number equations within sections (i.e. 1.1, 1.2, 2.1, 2.2 instead of 1, 2, 3, 4)
\numberwithin{figure}{section} % Number figures within sections (i.e. 1.1, 1.2, 2.1, 2.2 instead of 1, 2, 3, 4)
\numberwithin{table}{section} % Number tables within sections (i.e. 1.1, 1.2, 2.1, 2.2 instead of 1, 2, 3, 4)

\setlength\parindent{0pt} % Removes all indentation from paragraphs

\usepackage{enumitem} % Required for list customisation
\setlist{noitemsep} % No spacing between list items

%----------------------------------------------------------------------------------------
%	DOCUMENT MARGINS
%----------------------------------------------------------------------------------------

\usepackage{geometry} % Required for adjusting page dimensions and margins

\geometry{
	paper=a4paper, % Paper size, change to letterpaper for US letter size
	top=2.5cm, % Top margin
	bottom=3cm, % Bottom margin
	left=3cm, % Left margin
	right=3cm, % Right margin
	headheight=0.75cm, % Header height
	footskip=1.5cm, % Space from the bottom margin to the baseline of the footer
	headsep=0.75cm, % Space from the top margin to the baseline of the header
	%showframe, % Uncomment to show how the type block is set on the page
}
\setlength{\parskip}{\baselineskip}%
\setlength{\parindent}{0pt}%

%----------------------------------------------------------------------------------------
%	FONTS
%----------------------------------------------------------------------------------------

\usepackage[utf8]{inputenc} % Required for inputting international characters
\usepackage[T1]{fontenc} % Use 8-bit encoding

\usepackage{fourier} % Use the Adobe Utopia font for the document

%----------------------------------------------------------------------------------------
%	SECTION TITLES
%----------------------------------------------------------------------------------------

\usepackage{sectsty} % Allows customising section commands

\sectionfont{\vspace{6pt}\centering\normalfont\scshape} % \section{} styling
\subsectionfont{\normalfont\bfseries} % \subsection{} styling
\subsubsectionfont{\normalfont\itshape} % \subsubsection{} styling
\paragraphfont{\normalfont\scshape} % \paragraph{} styling

%----------------------------------------------------------------------------------------
%	HEADERS AND FOOTERS
%----------------------------------------------------------------------------------------

\usepackage{scrlayer-scrpage} % Required for customising headers and footers

\ohead*{} % Right header
\ihead*{} % Left header
\chead*{} % Centre header

\ofoot*{} % Right footer
\ifoot*{} % Left footer
\cfoot*{\pagemark} % Centre footer
 % Include the file specifying the document structure and custom commands

\title{
  \normalfont\normalsize
  \textsc{Universidade Federal de Pelotas, Ciência da Computação}\\ % Your university, school and/or department name(s)
  \vspace{25pt} % Whitespace
  \rule{\linewidth}{0.5pt}\\ % Thin top horizontal rule
  \vspace{20pt} % Whitespace
  {\huge Coléta e Análise de Dados Estatísticos}\\ % The assignment title
  \vspace{12pt} % Whitespace
  \rule{\linewidth}{2pt}\\ % Thick bottom horizontal rule
  \vspace{12pt} % Whitespace
}

\author{
  \LARGE Jonathas Conceição\\
  \LARGE Juan Rios
}

\date{\normalsize\today} % Today's date (\today) or a custom date

\begin{document}

\maketitle % Print the title

%----------------------------------------------------------------------------------------

\section{Introdução}

Para que possamos entender melhor a utilização da estatística em nosso dia a dia ou até mesmo em nossas respectivas áreas, é importante a realização de trabalhos que possam nos proporcionar tais demonstrações, visando compreender a sua importância e complexidade. Devido a isto, foi proposto como trabalho a realização de um levantamento estatístico por meio de um questionário, onde cada entrevistado deveria responder uma série de questões relacionadas ao curso vigente, a UFPel e até mesmo algumas questões um pouco mais pessoais, como o consumo de álcool.



bla bla bla os dados que escolhemos bla bla bla.

bla bla bla, objetivo de apresentar gráficos para visualização dos dados coletados.


Neste trabalho apresentamos então dois conjuntos de gráficos,
na seção \ref{sec:raw} apresentamos gráficos gerados diretamente dos dados coletados,
já na seção \ref{sec:cross}, nós apresentamos gráficos crusando algumas informações dos dados de entrada.

%----------------------------------------------------------------------------------------

\section{Gráficos diretos}\label{sec:raw}

bla bla bla, algum texto para não ficar seção seguida de subseção.

\subsection{Curso}

bla bla bla sobre info sobre a Figura \ref{fig:graph1}.

\begin{figure}[h]
  \centering
  \includegraphics[page=1, width=1.0\columnwidth]{Graphs.pdf}
  \label{fig:graph1}
  \caption{Distribuição do curso dos alunos}
\end{figure}

\subsection{Semestre de Ingresso}

bla bla bla, semestre e ano de ingresso na Figura \ref{fig:graph2}.

\begin{figure}[h]
  \centering
  \includegraphics[page=2, width=1.0\columnwidth]{Graphs.pdf}
  \label{fig:graph2}
  \caption{Ano/semestre de ingresso dos alunos}
\end{figure}

%----------------------------------------------------------------------------------------

\section{Gráficos cruzados}\label{sec:cross}

bla bla bla, algum texto sobre os gráficos cruzados na Figura \ref{fig:graph10}.

\subsection{Já pensou em desistir (por Curso)}

\begin{figure}[h]
  \centering
  \includegraphics[page=10, width=1.0\columnwidth]{Graphs.pdf}
  \label{fig:graph10}
  \caption{bla bla bla legenda}
\end{figure}

%----------------------------------------------------------------------------------------

\end{document}
