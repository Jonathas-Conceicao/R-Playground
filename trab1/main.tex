\documentclass[11pt]{scrartcl} % Font size
\input{structure.tex} % Include the file specifying the document structure and custom commands

\title{
  \normalfont\normalsize
  \textsc{Universidade Federal de Pelotas, Ciência da Computação}\\ % Your university, school and/or department name(s)
  \vspace{25pt} % Whitespace
  \rule{\linewidth}{0.5pt}\\ % Thin top horizontal rule
  \vspace{20pt} % Whitespace
  {\huge Coléta e Análise de Dados Estatísticos}\\ % The assignment title
  \vspace{12pt} % Whitespace
  \rule{\linewidth}{2pt}\\ % Thick bottom horizontal rule
  \vspace{12pt} % Whitespace
}

\author{
  \LARGE Jonathas Conceição\\
  \LARGE Juan Rios
}

\date{\normalsize\today} % Today's date (\today) or a custom date

\begin{document}

\maketitle % Print the title

%----------------------------------------------------------------------------------------

\section{Introdução}

Para que possamos entender melhor a utilização da estatística em nosso dia a dia ou até mesmo em nossas respectivas áreas, é importante a realização de trabalhos que possam nos proporcionar tais demonstrações, visando compreender a sua importância e complexidade. Devido a isto, foi proposto como trabalho a realização de um levantamento estatístico por meio de um questionário, onde cada entrevistado deveria responder uma série de questões relacionadas ao curso vigente, a UFPel e até mesmo algumas questões um pouco mais pessoais, como o consumo de álcool.



bla bla bla os dados que escolhemos bla bla bla.

bla bla bla, objetivo de apresentar gráficos para visualização dos dados coletados.


Neste trabalho apresentamos então dois conjuntos de gráficos,
na seção \ref{sec:raw} apresentamos gráficos gerados diretamente dos dados coletados,
já na seção \ref{sec:cross}, nós apresentamos gráficos crusando algumas informações dos dados de entrada.

%----------------------------------------------------------------------------------------

\section{Gráficos diretos}\label{sec:raw}

bla bla bla, algum texto para não ficar seção seguida de subseção.

\subsection{Curso}

bla bla bla sobre info sobre a Figura \ref{fig:graph1}.

\begin{figure}[h]
  \centering
  \includegraphics[page=1, width=1.0\columnwidth]{Graphs.pdf}
  \label{fig:graph1}
  \caption{Distribuição do curso dos alunos}
\end{figure}

\subsection{Semestre de Ingresso}

bla bla bla, semestre e ano de ingresso na Figura \ref{fig:graph2}.

\begin{figure}[h]
  \centering
  \includegraphics[page=2, width=1.0\columnwidth]{Graphs.pdf}
  \label{fig:graph2}
  \caption{Ano/semestre de ingresso dos alunos}
\end{figure}

%----------------------------------------------------------------------------------------

\section{Gráficos cruzados}\label{sec:cross}

bla bla bla, algum texto sobre os gráficos cruzados na Figura \ref{fig:graph10}.

\subsection{Já pensou em desistir (por Curso)}

\begin{figure}[h]
  \centering
  \includegraphics[page=10, width=1.0\columnwidth]{Graphs.pdf}
  \label{fig:graph10}
  \caption{bla bla bla legenda}
\end{figure}

%----------------------------------------------------------------------------------------

\end{document}
